People's political attitudes and values vary. Research over the last fifty years has suggested that this variation is due to differences in two foundational dimensions of political ideology, often labelled as economic and social conservatism, or social dominance and authoritarianism. While the existence of these two dimensions of political ideology is supported by evidence from political, social, moral, and cross-cultural psychology, less work has examined the essential nature of the two dimensions or asked why this particular two-dimensional structure organises political attitudes and values. In this thesis, I outline a dual evolutionary theoretical framework to understand the two dimensions of political ideology in humans. Synthesising and expanding on existing evolutionary approaches to politics, I argue that the two dimensions of political ideology have emerged from two key shifts in the evolution of human group living. First, humans began to cooperate more across wider interdependent networks and share the spoils of cooperation more evenly. Second, humans became more committed to group viability, conforming to social norms in culturally marked groups and punishing norm-violators. These key shifts correspond to economic and social conservatism, respectively. I test this theory by leveraging empirical tools from behavioural economics. Across several studies using abstract incentivised behavioural tasks, I show that cooperative and group conformist preferences reliably predict economically and socially conservative views. By supporting the dual evolutionary framework for political ideology, these results show how ancient social drives that evolved to help us navigate the challenges of human group living continue to shape the political landscape even today.