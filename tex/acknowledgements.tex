In the first year of my PhD, my supervisor Quentin Atkinson called me into his office to share a new paper he had found that tied in directly with our work on the evolution of political ideology. It was the perfect paper that would later form the bedrock of our theory. For me, it's that moment in particular --- the two of us standing in his office, enjoying the thrill of a new scientific leap --- that best characterises Quentin as a PhD supervisor. He treats his students as scientific equals, including them through every step of the research process and carefully considering all their ideas. Quentin's approach to supervision ensures that his students feel like valued members of a collaborative research team. Thanks Quentin, it's been a pleasure to work with you over the past three years.

I would also like to thank: my co-supervisors Ananish Chaudhuri and Chris Sibley, who have contributed extensively to the dual foundations theory outlined in this thesis; my regular lab mates (past and present) Amalia Bastos, Kyle Fischer, Romana Gruber, Rebecca Hassall, Thanos Kyritsis, Guy Lavender Forsyth, Patrick Neilands, Oliver Sheehan, and Tom Vardy, who have made it a joy to come into the office every day; my friends and flatmates Jen Hale, Gabriela Hern\`{a}ndez, Jasmine Khorasanee, Nicole Ramirez, and Julian Wilson, who survived several pandemic lockdowns with me; my partner Tamara Gussy, who has provided me with love and support and kept me sane over the last two years; and my father Neil Claessens and my brother Adam Claessens, who, along with the rest of my family, have encouraged me from the other side of the world. Without these people, I would never have been able to write this thesis.

Finally, I would like to acknowledge the funding provided by the University of Auckland. The Doctoral Scholarship has allowed me to spend three years of my life in one of the most beautiful countries in the world, an experience I will cherish for the rest of my life.
